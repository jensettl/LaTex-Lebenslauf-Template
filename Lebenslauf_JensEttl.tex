\documentclass{resume} % Use the custom resume.cls style

\usepackage[left=0.4 in,top=0.4in,right=0.4 in,bottom=0.4in]{geometry} % Document margins
\usepackage{ragged2e}
\usepackage{fontawesome}
\usepackage[english,ngerman]{babel}
\newcommand{\tab}[1]{\hspace{.2667\textwidth}\rlap{#1}} 
\newcommand{\itab}[1]{\hspace{0em}\rlap{#1}}
\newcommand{\linkedin}{\faLinkedin\hspace{0.2em}}
\newcommand{\github}{\faGithub\hspace{0.2em}}
\name{Jens Ettl} % Your name
\address{+49 176 78867038 \\ Karlsruhe, Deutschland \\ \today} 
\address{\href{mailto:ettljens@gmail.com}{ettljens@gmail.com} \\ \linkedin \href{https://www.linkedin.com/in/jens-ettl-807578211/}{www.linkedin.com/in/jens-ettl} \\ \github \href{https://github.com/xy}{github.com/jensettl} \\ \href{www.jensettl.com}{www.jensettl.com}
} 


\begin{document}

%----------------------------------------------------------------------------------------
%	EDUCATION SECTION
%----------------------------------------------------------------------------------------

\begin{rSection}{Bildungsweg}
    {\bf Bachelor Wirtschaftsinformatik}, Hochschule Karlsruhe - Technik und Wirtschaft \hfill {Erwartet 2024}
    \begin{itemize}
        \item Tutoriumsleitung und -organisation in den Modulen Datenbanken 1 und Statistik u. Business Intelligence.
        \item \raggedright{Wahlmodule: Erfolgreiche Durchführung von Data Science Projekten, Datenanalyse mit JMP, Machine Learning, User Centered Design}\\
    \end{itemize}

    {\bf Bachelor Informatik}, Karlsruhe Institut für Technologie \hfill {\textit{Abgebrochen} 2016 - 2018}

    {\bf Allgemeine Hochschulreife} \textit{Leistungsfächer Informatik, Mathematik und Physik} \hfill {2016}

\end{rSection}

%----------------------------------------------------------------------------------------
% TECHINICAL STRENGTHS	
%----------------------------------------------------------------------------------------
\begin{rSection}{Fähigkeiten und Fertigkeiten}

    \begin{tabular}{ @{} >{\bfseries}l @{\hspace{6ex}} l }
        Technical Skills & Programmiersprachen (Python, JavaScript), UML, BPMN, Adobe XD,                            \\
                         & SQL, Firebase, VisualStudioCode, Docker, Confluence, Jira, Microsoft Office               \\
        \\
        Soft Skills      & Teamwork, Kommunikation, Problemlösung, Strukturiertheit, Pünktlichkeit,                  \\
                         & Kreativität, Schnelle Auffassungsgabe                                                     \\
        \\
        Sonstiges        & Englisch \textit{(Verhandlungssicher)}, Französisch \textit{(vertiefte Grundkenntnisse)}, \\
                         & Führerschein A \& B
    \end{tabular}\\
\end{rSection}

%-----------------------------------------------------------------------------------------

\begin{rSection}{Berufserfahrung}

    \textbf{Productowner} \hfill Okt 2021 - März 2023\\
    CAS Software AG \hfill \textit{Karlsruhe, Deutschland}
    \begin{itemize}
        \itemsep -3pt {}
        \item \raggedright{Produktentwicklung unterstützen: Mitwirkung an der Konzeption und Entwicklung von Apps und MVPs, sowie Kenntnisse im Umgang mit CRM- und ERP-Systemen.}
        \item \raggedright{Kundenanforderungen umsetzen: Erfahrung in der direkten Arbeit mit Kundenanforderungen, inklusive Erstellung von User-Stories und Akzeptanzkriterien}
        \item \raggedright{Sprint-Planung und -Durchführung: Verantwortlich für die Planung und Durchführung von Sprint-Reviews und Retrospektiven}
    \end{itemize}

    \textbf{Vorstand der Fachschaft Wirtschaftsinformatik} \hfill Okt 2020 - März 2022\\
    Technische Hochschule Karlsruhe \hfill \textit{Karlsruuhe, Deutschland}
    \begin{itemize}
        \itemsep -3pt {}
        \item \raggedright{\textbf{Beziehungen und Interessenvertretung:} Generierung neuer Firmenbeziehungen, Vertretung der Studierendeninteressen gegenüber der Hochschulleitung und Mitwirkung bei politischen Entscheidungen.}
        \item \raggedright{\textbf{Teamentwicklung und Konfliktlösung:} Steigerung der Teamfähigkeit durch die Planung und Durchführung von Events und Workshops sowie das Management von Konflikten zwischen Studierenden und Dozenten.} 
        \item \raggedright{\textbf{Kommunikation und Dokumentation:} Kommunikation mit der Hochschulleitung, Erstellung von Präsentationen und Dokumentationen zur Unterstützung der Fachschaftsarbeit.}
    \end{itemize}
\end{rSection}

%----------------------------------------------------------------------------------------
\begin{rSection}{Zusätzliche Aktivitäten}
    \begin{itemize}
        \item 	Teilnahme an Workshops zu SCRUM und Qualitätsmanagement.
        \item	Mentor an der Hochschule für Erstisemestler und Quereinsteiger.
        \item   Sprachaufenthalt in England 2014
    \end{itemize}


\end{rSection}

%----------------------------------------------------------------------------------------
%	WORK EXPERIENCE SECTION
%----------------------------------------------------------------------------------------
\newpage
\begin{rSection}{Projekte}
    \vspace{-0.5em}
    \item \textbf{Automatisierung eines Kundenauftragsprozesses} \hfill {Februar 2022} \\
    \textit{(BPMN, Java, Camunda)} \hfill {3 Wochen, 4 Personen}\\
    \justifying{{Abbilden eines Kundenauftragsprozesses in BPMN und Untersuchung auf Automatisierungsmöglichkeiten.
    Anschließende Implementierung der Automatisierungsentwürfe als Gruppe und Simulation mit Camunda.}

    \item \textbf{Portfolio Website} \hfill {Januar 2024} \\
    \textit{(React, Tailwind, Firebase Hosting)} \hfill {1 Woche, 1 Person}\\ 
    {Entwurf, Entwicklung und Verwaltung meiner persönlichen Online-Visitenkarte zur Präsentation meiner Projekte, Fähigkeiten und
    Erfahrungen. \href{https://www.jensettl.com//}{(Online verfügbar unter www.jensettl.com)}}}

    \item \textbf{Identifizierung fehlerhafter Produkte} \hfill {März 2023} \\
    \textit{(Machine Learning, Pandas, Python)} \hfill {1 Woche, 1 Person}\\
    \justifying{Analyse eines umfassenden Datensatzes aus der Stahlproduktion mit Schwerpunkt auf Oberflächenunregelmäßigkeiten.
    Implementierung datengesteuerter Lösungen zur frühzeitigen Erkennung fehlerhafter Stahlprodukte
    unter Anwendung von Algorithmen des maschinellen Lernens.}

    \item \textbf{Untersuchung von Unternehmensdaten auf Compliance} \textit{(Datenanalyse, JMP)}\\
    {Untersuchung realer Unternehmensdaten mit Hilfe von statistischen Methoden, Entscheidungsbäumen und neuronalen Netzen,
    welche Arbeitsschritte in einem Produktionsprozess die Einhaltung der Vorschriften beeinflussen.}

    \item \textbf{gRPC Client-Server Kommunikation} \textit{(JavaScript, gRPC)}\\
    {Entwicklung einer effizienten Client-Server Kommunikation mit gRPC im Rahmen der Veranstaltung “Automatisierung von Geschäftsprozessen”.
    Dokumentation und Code wurde mit GitHub verwaltet. Aufgaben wurden mit Kanban in Trello organisiert.}

    \item \textbf{Git Workshop} \textit{(Git, Markdown)}\\
    {Entwicklung eines Git-Workshops für Studenten zum Erlernen der Grundlagen von Git und GitHub.
    Der Workshop fand an der Hochschule für Technik und Wirtschaft Karlsruhe statt.}

    \item \textbf{Automatisches Reporting System} \textit{(Python, Hosting)}\\
    {Konzeption und Implementierung eines automatisierten Reporting-Systems, das wöchentlich umfassende Daten zu verschiedenen Lebensbereichen wie Wetter und Aktienkurse generiert und diese per Email an mich sendet.}
    
    \item \textbf{Supply Chain Management Simulation} \textit{(Typescript, FastAPI, SCM)}\\ 
    {Ein Planspiel wurde durchgeführt, um eine Lieferkette in der Fahrradindustrie zu simulieren. Dann wurde eine vollständige Website entwickelt, um einen Planungszeitraum zu berechnen und zu simulieren.}
    
    \item \textbf{Modellierung eines IT-Modells} \textit{(Konzeption, Use-Cases, Persona)}\\ 
    {In der Veranstaltung "Modellierung von IT-Systemen" ging es um die Konzeption und Planung einer App mit Fokus auf den Kundenanforderungen, Use-Cases bis hin zur Abbildung der Prozesse mit BPMN und UML.}
    
    \item \textbf{Automatisiertes digitales Aufräumsystem} \textit{(Python, CRON)}\\
    {Automatisiertes Sortieren und Verschieben von Dateien in verschiedene Ordner basierend auf Dateityp und Erstellungsdatum.}
    
    \item \textbf{HEIC-JPEG Converter} \textit{(Python)}\\
    {Problestellung: HEIC Dateien können nicht auf allen Geräten geöffnet werden und Online-Konvertierer erfordern den Upload persönlicher Daten \\
    Lösung: Lokale Konvertierung von HEIC Dateien in das weit verbreitete JPEG Format.}
    
    \item \textbf{Nett-Hier Schnitzeljagd - Konzept}\\
    {Ausgestaltung einer Idee, eine Website zu entwicklen, die es ermöglicht, Fotos von Nett-Hier Sticker und deren Standort zu teilen. Als eine Art digitale Schnitzeljagd nach Stickern auf der ganzen Welt.}
\end{rSection}


%----------------------------------------------------------------------------------------
\begin{rSection}{Auszeichnungen und Zertifikate}
    \begin{itemize}
        \item \textbf{Arbeitszeugnis}, CAS Software AG
        \item \textbf{Modern JavaScript}, Udemy
        \item \textbf{Grundlagen des Onlinemarketings}, Google Zukunftswerkstatt
        \item \textbf{RPA Starter}, UI Path
    \end{itemize}


\end{rSection}


\end{document}
